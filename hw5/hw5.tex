% --------------------------------------------------------------
% This is all preamble stuff that you don't have to worry about.
% Head down to where it says "Start here"
% --------------------------------------------------------------
 
\documentclass[12pt]{article}
 
\usepackage[margin=1in]{geometry} 
\usepackage{amsmath,amsthm,amssymb}
 
\newcommand{\N}{\mathbb{N}}
\newcommand{\Z}{\mathbb{Z}}
 
\newenvironment{theorem}[2][Theorem]{\begin{trivlist}
\item[\hskip \labelsep {\bfseries #1}\hskip \labelsep {\bfseries #2.}]}{\end{trivlist}}

\newenvironment{lemma}[2][Lemma]{\begin{trivlist}
\item[\hskip \labelsep {\bfseries #1}\hskip \labelsep {\bfseries #2.}]}{\end{trivlist}}

\newenvironment{exercise}[2][Exercise]{\begin{trivlist}
\item[\hskip \labelsep {\bfseries #1}\hskip \labelsep {\bfseries #2.}]}{\end{trivlist}}

\newenvironment{problem}[2][Problem]{\begin{trivlist}
\item[\hskip \labelsep {\bfseries #1}\hskip \labelsep {\bfseries #2}]}{\end{trivlist}}

\newenvironment{question}[2][Question]{\begin{trivlist}
\item[\hskip \labelsep {\bfseries #1}\hskip \labelsep {\bfseries #2.}]}{\end{trivlist}}

\newenvironment{corollary}[2][Corollary]{\begin{trivlist}
\item[\hskip \labelsep {\bfseries #1}\hskip \labelsep {\bfseries #2.}]}{\end{trivlist}}

\usepackage{graphicx}
\graphicspath{{./}}

\begin{document}
 
% --------------------------------------------------------------
%                         Start here
% --------------------------------------------------------------
 
\title{Homework 5}%replace X with the appropriate number
\author{Yunzhong He\\ %replace with your name
204010749} %if necessary, replace with your course title
 
\maketitle
 
\begin{problem}{1. Bias-Variance Tradeoff}
\def \yy {X^T\hat{\beta_\lambda}}
\def \ys {X^T\beta^*}

\item{(a)}
To minimize $\hat{\beta_\lambda}$ we have
\begin{align*}
	\frac{d (y-X\beta)'(y-X\beta)}{d \beta} = -2X'y + 2(X'X)^{-1}\beta + 2\lambda\beta = 0
\end{align*}
Thus
\begin{align*}
	\hat{\beta_\lambda} = (X'X - \lambda I)^{-1}X'y
\end{align*}
Plug in $y = \ys + \epsilon$ we have
\begin{align*}
	\hat{\beta_\lambda} = (X'X - \lambda I)^{-1}X'(X'\beta^* + \epsilon) = A(\lambda)b + A(\lambda)\epsilon
\end{align*}
where A = $(X'X - \lambda I)^{-1}X'$ and b = $X'\beta^*$
\\\\
\item{(b)} The bias is thus
\begin{align*}
	E[\yy] - \ys = A(\lambda)b + A(\lambda)E(\epsilon) - \ys
\end{align*}
\\
\item{(c)} The variance is thus
\begin{align*}
	E[(\yy - E(\yy))^2] = (X'A(\lambda))^2 var(\epsilon)
\end{align*}
\item{(d)} From (b) (c) and bias-variance theorem we can see that when $\lambda$ is large, bias dominates the squred errors, and when $\lambda$ is small, bias dominates squared error. 
\end{problem}

\begin{problem}{2. Kernalized Perception}
\item{(a)} Proof by induction:\\
(1) At iteration T = 0, we have $w = 0 = \sum_i^N 0\cdot x_i$, and is a linear combination of $x$.\\
(2) Assume at iteration T = k, $w$ is a linear combination of $x$, then at T = k+1 we have 
\begin{align*}
&w_{k+1} = w_k + \phi(x_{k+1}) \ if \ w'\phi(x) > 0\\
&w_{k+1} = w_k - \phi(x_{k+1}) \ otherwise
\end{align*}
In either case $w_{k+1}$ is a linear combination of $\phi(x)$, thus $w$ is always a linear combination of $\phi(x)$.
\\
\item{(b)} From the proof in (a) we can see that $w_k$ is actually a linear combination of $\phi(x_1), \phi(x_2), ..., \phi(x_k)$, and the prediction rule at T = k+1 can be written as
\begin{align*}
	y &= sign(w_k'\phi(x_{k+1})) = sign(\sum_i^{k} \alpha_i\phi(x_i)'\phi(x_{k+1})) \\
	  &= sign(\sum_i^{k} \alpha_i<\phi(x_i), \phi(x_{k+1})>)
\end{align*}
which only depends on the inner products of the mapping $\phi$.
\\
\item{(c)}
From (b) we know that the perception algorithm can be rewritten as follows.\\
(1) Initialize $\alpha_0 = 0$\\
(2) At iteration T = k+1, compute $y$ and update $w$ as follows\\
\indent val = 0\\
\indent For i from 1 to k\\
\indent \indent val = val + $\alpha_i <\phi(x_i), \phi(x_{k+1})>$\\
\indent end\\
\indent $y = sign(val)$\\
\indent $w = w + y\phi(x_{k+1})$\\
\end{problem}

\begin{problem}{3. Kernels}
\item{(a)}
Since $K_1$ $K_2$ are positive semi-definite, we have $x^TK_1x$ and $x^TK_2x$ being non-negative at every entry for any vector $x$. Therefore we have
\begin{align*}
x^TK_3x = x^T(a_1K_1 + a_2K_2)x = a_1x^TK_1x + a_2x^TK_2x
\end{align*}
Since $a_1 \geq 0$ and $a_2 \geq 0$, we know every entry in $x^TK_3x$ is non-negative as well. Thus $K_3$ is positive semi-definite, and by Mercer's theorm so is $k_3$.
\\
\item{(b)} 
For every entry $K_4(i, j)$ of $K_4$, we know
\begin{align*}
K_4(i, j) = f(x_i)f(x_j) = f(x_j)f(x_i) = K_4(j, i)
\end{align*}
Thus $K_4$ is a symmetric matrix and is therefore positive semi-definite, by Mercer's theorm so is $k_4$
\item{(c)}
For any vector $x$, suppose $u = x^TK_1x$ and $v = x^TK_2x$, we know every entries in $u$ and $v$ are non-negative. Also, for $w = x^TK_5x$, we know $x(i) = u(i)v(i)$ because $k_5(x, x') = k_1(x, x')k_2(x, x')$. Thus every entries in $w$ are also non-negative, and $K_5$ is positive semi-definite. By Mercer's theorm $k_5$ is postitive semi-definite.
\end{problem}

\begin{problem}{4. Soft Margin Hyperplanes}
\item{(a)} With more general form of soft margin $\xi_i^p$, the dual formulation of optimization objective is thus
\begin{align*}
	&min_{\xi^p, w, b} \frac{1}{2}||w||_2^2 + C\sum_n||\xi_n^p||_2^2 \ s.t.\\
	&\forall n \ y_n[w^T\phi(x_n)+b] \geq 1 - \xi_n^p, \ \ \xi_n^p \geq 0, \ \ p > 1
\end{align*}
\item{(b)}
This more general formulation allows extra degrees of freedom at different dimensions in the choice of $\xi$, so it is able to generate more flexible margins. However, each $\xi_n^p$ now is a p-dimentional vector rather than a real number, making the optimization objective more complex.
\end{problem}

\begin{problem}{5. Programming}
\item{5.1}
The 10th feature's mean is \textbf{2.527} and standard deviation is \textbf{1.1245}. We estimate the mean and standard deviation from training data because the goal is to normalize the training data, and it does not matter whether it is a good estimate to the real distribution or not.
\item{5.3 (a)}
The cross-validation accuracy for different C can be found below
\begin{center}
	C     \:\: accu\\
	$4^{-6}$ \: 0.659\\
	$4^{-5}$ \: 0.659\\
	$4^{-4}$ \: 0.659\\
	$4^{-3}$ \: 0.659\\
	$4^{-2}$ \: 0.673\\
	$4^{-1}$ \: 0.725\\
	$4^{-0}$ \: 0.802\\
	$4^1$ \:\:\:  0.837\\
	$4^2$ \:\:\:  0.841\\
\end{center}
\item{5.3 (b) (c)}
Based on 5.3 we shoud use \textbf{C = 16}, and the corresponding test accuracy is \textbf{0.852}
\end{problem}
% --------------------------------------------------------------
%     You don't have to mess with anything below this line.
% --------------------------------------------------------------
 
\end{document}
