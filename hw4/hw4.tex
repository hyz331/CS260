% --------------------------------------------------------------
% This is all preamble stuff that you don't have to worry about.
% Head down to where it says "Start here"
% --------------------------------------------------------------
 
\documentclass[12pt]{article}
 
\usepackage[margin=1in]{geometry} 
\usepackage{amsmath,amsthm,amssymb}
 
\newcommand{\N}{\mathbb{N}}
\newcommand{\Z}{\mathbb{Z}}
 
\newenvironment{theorem}[2][Theorem]{\begin{trivlist}
\item[\hskip \labelsep {\bfseries #1}\hskip \labelsep {\bfseries #2.}]}{\end{trivlist}}

\newenvironment{lemma}[2][Lemma]{\begin{trivlist}
\item[\hskip \labelsep {\bfseries #1}\hskip \labelsep {\bfseries #2.}]}{\end{trivlist}}

\newenvironment{exercise}[2][Exercise]{\begin{trivlist}
\item[\hskip \labelsep {\bfseries #1}\hskip \labelsep {\bfseries #2.}]}{\end{trivlist}}

\newenvironment{problem}[2][Problem]{\begin{trivlist}
\item[\hskip \labelsep {\bfseries #1}\hskip \labelsep {\bfseries #2}]}{\end{trivlist}}

\newenvironment{question}[2][Question]{\begin{trivlist}
\item[\hskip \labelsep {\bfseries #1}\hskip \labelsep {\bfseries #2.}]}{\end{trivlist}}

\newenvironment{corollary}[2][Corollary]{\begin{trivlist}
\item[\hskip \labelsep {\bfseries #1}\hskip \labelsep {\bfseries #2.}]}{\end{trivlist}}

\usepackage{graphicx}
\graphicspath{{./}}

\begin{document}
 
% --------------------------------------------------------------
%                         Start here
% --------------------------------------------------------------
 
\title{Homework 4}%replace X with the appropriate number
\author{Yunzhong He\\ %replace with your name
204010749} %if necessary, replace with your course title
 
\maketitle
 
\begin{problem}{1. Liner Regression with Heterogenous Noise}
\item{a.}
Assume independent Gaussian distribution of $\epsilon_n$, the log likelihood density is given by
\begin{align*}
		logp(D|\beta, \sigma) &= log\prod_iP(x_i, y_i| \beta, \sigma) = log\prod N(0, \sigma_i)(\epsilon_i) \\
		&= log\prod_i \frac{1}{\sigma_i^2 \sqrt{2\pi}} \cdot exp(\sum \frac{-\epsilon^2}{2\sigma_i^2}) = -\sum_i(\frac{\epsilon_i^2}{2\sigma_i^2}-log\frac{1}{2\sigma_i^2}) \\
		&= -\sum_i(\frac{(x_i^T\beta - y_i)^2}{2\sigma_i^2}-log\frac{1}{2\sigma_i^2}) 
\end{align*}
\item{b.}
From (a) we can see that maximizing $logp(D|\beta,\alpha)$ is essentially minimizing $\sum_i\frac{(x_i^T\beta - y_i)^2}{2\sigma_i^2}$ because we can ignore all constant terms. So let
\begin{align*}
    S = \begin{bmatrix}
			- x_1/\sigma_1 -\\ 
			- x_2/\sigma_2 -\\ 
			      :         \\
			- x_n/\sigma_n -\\ 
		\end{bmatrix}
	\ \ \
    Y = \begin{bmatrix}
			y_1\\
			y_2\\
			:\\
			y_n\\
		\end{bmatrix}
\end{align*}
Then for $\beta^*$ that minimizes it we have the normal equation
\begin{align*}
	S^TS\beta = S^TY
\end{align*}
So the optimal $\beta^*$ is thus
\begin{align*}
	\beta^* = (S^TS)^{-1}S^TY
\end{align*}
\end{problem}

\begin{problem}{2. Linear Regression with Smooth Coeficients}
\item{a.}
Adding a smooth coeficient we have the follwing loss function
\begin{align*}
		L(\beta) = \sum_i (x_i^T\beta - y_i)^2 + \frac{\sum_i (\beta_{i+1} - \beta_i)^2}{p} + \lambda ||\beta||^2
\end{align*}
\def \diffm 
{\begin{bmatrix}
		0 & 1 & 0 & ... & 0 & 0\\	
		-1& 0 & 1 & ... & 0 & 0\\	
		0 & -1 & 0 & ... & 0 & 0\\	
		: & : & :  & ... & : & :\\	
		0 & 0 & 0  & ... & -1 & 0\\	
\end{bmatrix}}
\def \bigx
{\begin{bmatrix}
	 -x_1-\\
	 -x_2-\\
	  : \\
	 -x_n-\\
\end{bmatrix}}
\item{b}
Taking partial derivatives of $L(\beta)$ we have
\begin{align*}
	\frac{\partial L(\beta)}{\partial \beta_j} = 2\sum_i (x_i^T \beta x_{ij} - y_ix_{ij}) + \frac{2}{p} (\beta_{j+1} - \beta_{j-1}) + \lambda 2\beta_j
\end{align*}
In matrix form we have
\begin{align*}
	\frac{\partial L(\beta)}{\partial \beta} = 2(X^2\beta - Xy + \frac{1}{p} M \beta + \lambda\beta)
\end{align*}
where
\begin{align*}
	M = \diffm \ \ \ X = \bigx
\end{align*}
Thus for the optimal $\beta^*$ we have
\begin{align*}
	X^2\beta^* - Xy + \frac{1}{p} M \beta^* + \lambda\beta^* = \textbf{0}
\end{align*}
And 
\begin{align*}
	\beta^* = (\bigx^2 + \frac{1}{p}\diffm + \lambda I)^{-1}\bigx y
\end{align*}
\end{problem}

\begin{problem}{3. Linearly Constrained Linear Regression}
\item{}
Add a linearly constrain into normal equation $X^TX\beta = X^Ty$, we have the following system of equations
\begin{align*}
	&X^TX\beta = X^Ty\\
	&A\beta = b
\end{align*}
In matrix form, we have
\begin{align*}
\begin{bmatrix}
	X^TX \\
	A
\end{bmatrix}
\beta = 
\begin{bmatrix}
	X^Ty \\
	b
\end{bmatrix}
\end{align*}
Thus 
\begin{align*}
\beta = 
\begin{bmatrix}
	X^TX \\
	A
\end{bmatrix}^{-1}
\begin{bmatrix}
	X^Ty \\
	b
\end{bmatrix}
\end{align*}
\end{problem}

\begin{problem}{4. Onine Learning}
	For update rule $w^{t+1} = w^t + k$, we want to find the minimal $k$ under the following constrain 
\begin{align*}
	& (w^t+k)^Tx_i > 0 \ if \ y_i = +1\\
	& (w^t+k)^Tx_i < 0 \ if \ y_i = -1
\end{align*}
Since by perception learing algorithm, the update won't happen if $k=0$ satisfies the constrain, the minimal solution always happens on the boundry $(w^t + k)^Tx_i = 0$.
Solving the least norm solution to this system we have $k = -X(X^TX)^{-1}X^Tw$, in which $X$ is a row vector of $x_i$.
\end{problem}

% --------------------------------------------------------------
%     You don't have to mess with anything below this line.
% --------------------------------------------------------------
 
\end{document}
